\chapter{Architecture and Design}
This chapter describes a little bit about the overview of the plugin, the programming language that was used, how the plugin is deployed, what it needs in order to be executed successfully etc. It also describes why that specific programming language and the libraries were chosen and the main reasons sitting behind those decisions.\\
\hyperref[sec:ThePlugin]{The Plugin} section of this chapter describes the plugin in a more-depth manner, some of the libraries that were used, how it is bundled together and the components that are crucial to the plugin.
\section{Overview}
The plugin is developed and built using the \href{https://www.java.com/en/}{Java} programming language (Java Platform SE 8), which is fully compliant with the JOSM development environment (JOSM also supports higher Java versions). The plugin is developed as a "separate" component from JOSM, which can be manually added/installed to an existing JOSM application environment.\\
\newline
The plugin is ultimately deployed as a single \textit{.jar} file, which contains all the necessary components, dependencies and libraries that the plugin needs in order to execute successfully. This jar file however, does not serve its purpose if executed solely, it requires a JOSM application environment and must be executed from it in order for it to serve its purpose. It is a relatively heavy .jar file because of the XML components that it has bundled within, which boil down to a Java Model of the whole NeTEx XML schema containing all its components, their necessary interrelations and methods.

\section{Design Decision - Programming Language}
This plugin was built using the Java programming language, version 8.\\
\newline
\href{https://www.java.com/en/}{Java} is a powerful general-purpose programming language. It is used to develop desktop and mobile applications, big data processing, embedded systems, and so on. According to Oracle, the company that owns Java, Java runs on 3 billion devices worldwide, which makes Java one of the most popular programming languages. \cite{WhatIsJava}\\
\newline
The reason why this plugin was chosen to be built and developed using Java is because of JOSM, it runs on Java, so everything under it must run on Java too, including the plugins. All of the JOSM plugins are developed inside the JOSM environment as separate .jar files and must use JOSM native Java methods in order to be executed and incorporated within JOSM.\\
It is also an Object-Oriented programming language, which runs under ANY operating system (because of it's compiler) and uses inheritance and abstract methods (Object Oriented programming principles), which turned out to be very useful and neat for the nature of this plugin.\\
\newline
Here's a snippet of Java code \& syntax example of a Java Program that computes the quotient and the remainder of a division: \cite{JavaExampleQuadraticEquation}
\begin{minted}{java}
public class QuotientRemainder {
	
	public static void main(String[] args) {
		
		int dividend = 25, divisor = 4;
		
		int quotient = dividend / divisor;
		int remainder = dividend % divisor;
		
		System.out.println("Quotient = " + quotient);
		System.out.println("Remainder = " + remainder);
	}
}
\end{minted}
\newpage
\section{The Plugin}
\label{sec:ThePlugin}
\lipsum[8-9]

